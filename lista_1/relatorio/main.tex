\documentclass[a4paper,12pt]{article}
\usepackage[utf8]{inputenc}
\usepackage[alf]{abntex2cite}
\usepackage[english]{babel}
\usepackage{amsmath} % Para equações matemáticas
\usepackage{geometry}
\usepackage{cite}
\geometry{margin=1in}
\usepackage{hyperref}
\usepackage{natbib}
\usepackage{parskip}
\usepackage{longtable}
\usepackage{xcolor}
\usepackage{tabularx}
\usepackage[colorinlistoftodos,prependcaption,textsize=tiny]{todonotes}
\usepackage{comment}
\usepackage{graphicx} % Para resizebox
\usepackage{array} % Para colunas com largura fixa
\usepackage{listings}
\usepackage{siunitx} % For \si{\micro\second}
\usepackage[acronym,nonumberlist]{glossaries}

% Configuring listings for code and log display
\lstset{
  basicstyle=\ttfamily\small,
  breaklines=true,
  breakatwhitespace=true,
  frame=single,
  numbers=left,
  numberstyle=\tiny,
  keywordstyle=\color{blue},
  commentstyle=\color{gray},
  showstringspaces=false,
  tabsize=2,
}


\title{Lista 1 - Modelagem do Conhecimento em Lógica }
\author{André Thiago Pfleger \\  Gustavo Girotto \\ João Pedro Schmidt Cordeiro}
\date{\today}

\begin{document}

\begin{titlepage}
   \begin{center}
      \textsc{Universidade Federal de Santa Catarina} \\
      \textsc{Centro Tecnológico} \\
      \textsc{Departamento de Informática e Estatística} \\[4cm]

      \textbf{\Large{INE5430 - Inteligência Artificial}} \\[2cm]

      \textbf{\huge{Lista 1 - Modelagem do Conhecimento em Lógica }} \\[3cm]

      André Thiago Pfleger \\
      Gustavo Girotto \\
      João Pedro Schmidt Cordeiro \\
      
      \vfill
      
      {Florianópolis} \\
      {\today}
   \end{center}
\end{titlepage}

\section{Modelagem do Conhecimento em Lógica}
\subsection{Questão 1}
\section{Variedades (descrição, torra, moagem e preparo)}
A seguir, para cada variedade há: descrição resumida, torra recomendada (e justificativa), moagem sugerida e modos de preparo indicados.

\subsection{Robusta (Coffea canephora)}
{Descrição} Sabor acentuado e rústico, corpo pesado, maior resistência a doenças e teor de cafeína elevado; frequentemente utilizado em blends e produtos solúveis.  
{Torra recomendada} \textbf{Média a escura} — realça corpo e notas torradas, equilibrando a acidez naturalmente baixa.  
{Moagem sugerida} Fina (espresso) a média (filtros rápidos), conforme o método.  
{Modos de preparo indicados} Espresso, moka, blends industriais e cafés instantâneos.

\subsection{Conilon (grupo canephora, termo comum no Brasil)}
{Descrição} Variante de canephora muito explorada no Brasil; grãos tipicamente menores, amargor pronunciado e alto teor de cafeína; uso comum em commodity e blends.  
{Torra recomendada} \textbf{Média a escura} — enfatiza corpo e notas de torra.  
{Moagem sugerida} Fina (espresso) / média (filtrados comerciais).  
{Modos de preparo indicados} Espresso, blends e solúveis.

\subsection{Arábica (Coffea arabica — genérico)}
{Descrição} A variedade mais consumida globalmente; perfis sensoriais complexos com acidez, doçura e notas frutadas/ florais, fortemente dependentes de terroir e altitude.  
{Torra recomendada} \textbf{Clara a média} — preserva as características de origem e a acidez desejada.  
{Moagem sugerida} Média a média-fina (pour-over); fina para espresso specialty.  
{Modos de preparo indicados} Pour-over (V60), Chemex, Aeropress, filtros e espresso specialty.

\subsection{Bourbon}
{Descrição} Subgrupo de Arábica com doçura pronunciada e aroma intenso; bebida tipicamente suave.  
{Torra recomendada} \textbf{Clara a média} — evidencia doçura e nuances carameladas/frutadas.  
{Moagem sugerida} Média a média-fina.  
{Modos de preparo indicados} Pour-over, Chemex, Aeropress; prensa francesa quando se busca maior corpo.

\subsection{Catuaí (Catuai)}
{Descrição} Cultivar brasileira resultante de cruzamentos; planta compacta, produtividade e doçura natural; amplamente plantado.  
{Torra recomendada} \textbf{Média} — equilíbrio entre doçura e corpo.  
{Moagem sugerida} Média.  
{Modos de preparo indicados} Filtrados (V60, coador), blends para espresso, prensa francesa.

\subsection{Mundo Novo (Novo Mundo)}
{Descrição} Cruzamento natural (Typica × Bourbon); vigoroso, produtivo; tende a oferecer corpo mais pronunciado e doçura.  
{Torra recomendada} \textbf{Média a média-escura} — realça corpo e notas doces.  
{Moagem sugerida} Média a média-grosseira.  
{Modos de preparo indicados} Pour-over, prensa francesa, blends que visem corpo.

\subsection{Caturra}
{Descrição} Mutação do Bourbon com planta de porte compacto; boa produtividade inicial; produz cafés de boa qualidade e doçura.  
{Torra recomendada} \textbf{Clara a média} — preserva equilíbrio entre acidez e doçura.  
{Moagem sugerida} Média.  
{Modos de preparo indicados} Filtrados, Chemex, Aeropress, single-origin.

\subsection{Acaiá (Acaia)}
{Descrição} Mutação do Mundo Novo; variedade rara, com bom desempenho em altitudes acima de \(\sim\)800\,m; xícara com notas frutadas e lembranças de chocolate.  
{Torra recomendada} \textbf{Clara a média} — evidencia notas frutadas e acidez média.  
{Moagem sugerida} Média-fina (pour-over) / média para métodos com mais contato.  
{Modos de preparo indicados} Pour-over, Aeropress, cupping.

\subsection{Typica}
{Descrição} Variedade histórica e clássica; perfil doce e floral/complexo; menor produtividade e maior sensibilidade a pragas.  
{Torra recomendada} \textbf{Clara} — preserva delicadeza e complexidade aromática.  
{Moagem sugerida} Média-fina.  
{Modos de preparo indicados} Pour-over, V60, cupping.

\subsection{Geisha (Gesha)}
{Descrição} Variedade muito expressiva aromaticamente (jasmim, flores, frutas de caroço); extremamente valorizada no mercado specialty.  
{Torra recomendada} \textbf{Clara} (por vezes muito clara) — para não mascarar notas voláteis e florais.  
{Moagem sugerida} Média-fina a média.  
{Modos de preparo indicados} Pour-over (V60), Kalita/Wave, métodos que priorizem aroma e limpeza.

\subsection{Maragogipe}
{Descrição} Mutação de Typica com sementes excepcionalmente grandes ("grão-elefante"); perfil muitas vezes suave e frutado; baixa produtividade.  
{Torra recomendada} \textbf{Clara a média} — destaca notas delicadas.  
{Moagem sugerida} Média.  
{Modos de preparo indicados} Filtrados especiais, single-origin.

\subsection{Pacamara (Pacas × Maragogipe)}
{Descrição} Híbrido que produz grãos grandes e perfil complexo (floral, frutado e corpo médio).  
{Torra recomendada} \textbf{Clara a média} — realça complexidade sem sobrecarregar com notas de torra.  
{Moagem sugerida} Média-fina a média.  
{Modos de preparo indicados} Pour-over, Aeropress, extrações experimentais em pequenos lotes.

\section{Recomendações práticas gerais}
\begin{itemize}
  \item \textbf{Moagem por método (resumo):} Espresso: \emph{fina}; Aeropress/V60: \emph{média-fina}; Chemex: \emph{média-grosseira}; French press: \emph{grossa}; Cold brew: \emph{extra grossa}.
  \item \textbf{Torra:} Clara preserva origem e acidez; média equilibra doçura e corpo; escura traz notas de torra e reduz percepção de acidez — escolha conforme objetivo sensorial.
  \item \textbf{Ajustes (dial-in):} Afine moagem, dose e tempo/temperatura antes de alterar torra; pequenas mudanças na moagem provocam grandes mudanças na extração.
\end{itemize}

% construindo a tabela com base nas descrições
\section{Resumo das Variedades em Tabela}

\begin{table}[h!]
\centering
\caption{Resumo das características das variedades de café}
\begin{tabular}{|>{\raggedright\arraybackslash}p{3.5cm}|>{\raggedright\arraybackslash}p{2.5cm}|>{\raggedright\arraybackslash}p{2.5cm}|>{\raggedright\arraybackslash}p{4cm}|}
\hline
\textbf{Variedade} & \textbf{Torra Recomendada} & \textbf{Moagem} & \textbf{Métodos de Preparo} \\
\hline
Robusta & Média a escura & Fina a média & Espresso, moka, blends, instantâneo \\
\hline
Conilon & Média a escura & Fina a média & Espresso, blends, solúvel \\
\hline
Arábica & Clara a média & Média a fina & Pour-over, Chemex, Aeropress, espresso \\
\hline
Bourbon & Clara a média & Média a fina & Pour-over, prensa francesa \\
\hline
Catuaí & Média & Média & V60, blends, prensa francesa \\
\hline
Mundo Novo & Média a escura & Média a grossa & Pour-over, blends, prensa \\
\hline
Caturra & Clara a média & Média & Filtrados, Aeropress \\
\hline
Acaiá & Clara a média & Média a fina & Pour-over, cupping \\
\hline
Typica & Clara & Média a fina & V60, cupping \\
\hline
Geisha & Clara & Média a fina & Pour-over, Kalita \\
\hline
Maragogipe & Clara a média & Média & Filtrados especiais \\
\hline
Pacamara & Clara a média & Média a fina & Pour-over, Aeropress \\
\hline
\end{tabular}
\end{table}


\bibliographystyle{IEEEtran}
\bibliography{references}
\nocite{counterculture_pourover}
\nocite{embrapa_catalog}
\nocite{perfectdailygrind_geisha}
\nocite{sca_brewing_chart}
\nocite{sca_standard_pdf}
\nocite{wcr_varieties}
\nocite{wikipedia_geisha}

\end{document}
